\documentclass[12pt]{article}
\usepackage[top=1in, bottom=1in, left=1in, right=1in]{geometry}
\usepackage{hyperref}
\usepackage{ifpdf}
\usepackage{csquotes}
\usepackage[american]{babel}
\usepackage[backend=bibtex,bibstyle=alphabetic,citestyle=reading,autocite=plain]{biblatex}
\addbibresource{bibliography.bib}

\title{Code and Coffee General Career Guide}
\author{Code and Coffee}
\date{\today}

\begin{document}

\maketitle

\section{Common Technical Skills}
\begin{itemize}
\item Know the UNIX shell
\item Learn a CLI editor (vim, Nano, Emacs, etc)
\item Know at least one flavor of SQL
\item Know at least one source control (IE Git)
\item Know at least one CI/CD (IE Jenkins/CircleCI/Travis)
\end{itemize}

\section{Core Software Development}
\begin{itemize}
\item \autocite{mcdowell2015interview}
\item \autocite{skiena2008algorithm}
\item \autocite{gamma1994patterns}
\item \autocite{martin2010clean}
\item Learn the basics of Agile
\item A ticketing system like Jira or Trello
\item Know about testing.  For example, unit testing and integration testing
\end{itemize}


\section{Getting the Job}
\begin{itemize}
\item Research positions way in advance before applying (Position requirements, interview questions).  Don't just starting flinging out resume's after you graduate.
\item Research the specific company you are applying to
\item Balance what's in demand vs what you like to do
\item Find examples of resumes and research how to make a resume.  Get yours reviewed by multiple people with work experience
\item Set up a Linked in
\item Set up a \textit{professional} website
\item Have an online Git and make sure to show off only your best work.  Most interviewers give two minutes to look at your profile
\item Make a professional business card
\item Know the 6 month time limit.  It looks bad if it you have more than 6 months without work.  Best case case scenario it will take you 2--3 months to get a job
\item November through February is a slow hiring season
\item Network.  Professional networking with people is important
\item Always be working on something
\item Practice mock interviews and whiteboard coding interviews
\end{itemize}

\section{Career Advice}
\begin{itemize}
\item Watch out for recruiters
  \begin{itemize}
  \item Watch out for 3rd party recruiters
    \begin{itemize}
    \item They sell data to third parties
    \item They bait and switch
    \item They take commission (18\%-30\%) which lowers your pay negotiation
    \end{itemize}
  \item They are not programmers and do not know technical details (they don't typically know the difference between C and C++)
  \item Watch out for foreign agencies
  \item Make your salary requirements clear and firm
  \item {\em Be nice to them}, they can get you jobs and you also have a professional reputation to maintain
  \end{itemize}
\item Never stop researching.  You should still have side projects after getting in.
\item Watch for IP agreements in your contract
\item Get advice from more senior engineers
\item Get contract jobs reviewed by a lawyer
\item Go to hackerspaces and meetups
\item Document issues with difficult coworkers
  \begin{itemize}
  \item Keep screenshots of messages
  \item Get every promise and ``verbal contract'' in written form
  \end{itemize}
\item Make sure you're paperwork is in order
  \begin{itemize}
  \item Get your 1099 or W-2 for contract and employment work, respectively
  \item Keep good financial records for when you need to file taxes, etc.
  \item If contracting, make sure to set aside money for periods of unemployment and have a ``rainy day'' fund.  In addition to setting aside part of each payment to pay the taxes on that sum
  \end{itemize}
\item Go to conferences.  For example:
  \begin{itemize}
  \item SCaLE, Southern California Linux Expo: Annual Linux Conference and Expo in Pasadena, California
  \item AWS re:invent: Annual AWS DevOps tools expo/conference in Las Vegas, Nevada
  \item Strange Loop: Annual conference focused on advances in professional software development in St. Louis, Missouri
  \item Grace Hopper Celebration of Women in Computing
  \item Layer 1: Annual Hacker and Security Conference in Los Angeles, California
  \item DEF CON: Annual Hacker and Security Conference in Las Vegas, Nevada
  \item Black Hat: Annual Security Conference in Las Vegas, Nevada
  \item HashiConf: Annual Expo for HashiCorp Products and DevOps tools
    % \item SparkleCon
  \item DevOps Summit
  \item PyCon: Python Conference
  \item RustConf: Rust developers conference
  \item GDC, Game Developer Conference: Annual game development conference focusing on professionals in the industry
  \item IndieCade: Indie game developer conference
  \end{itemize}
\end{itemize}

\section{Communication Skills Matter}
\begin{itemize}
\item Establishing Focus: The ability to develop and communicate goals in support of the business’ mission.
\item Providing Motivational Support: The ability to enhance others’ commitment to their work.
\item Fostering Teamwork: As a team member, the ability and desire to work cooperatively with others on a team; as a team leader, the ability to demonstrate interest, skill, and success in getting groups to learn to work together.
\item Empowering Others: The ability to convey confidence in employees’ ability to be successful, especially at challenging new tasks; delegating significant responsibility and authority; allowing employees freedom to decide how they will accomplish their goals and resolve issues.
\item Managing Change: The ability to demonstrate support for innovation and for organizational changes needed to improve the organization’s effectiveness; initiating, sponsoring, and implementing organizational change; helping others to successfully manage organizational change.
\item Developing Others: The ability to delegate responsibility and to work with others and coach them to develop their capabilities.
\item Managing Performance: The ability to take responsibility for one’s own or one’s employees’ performance, by setting clear goals and expectations, tracking progress against the goals, ensuring feedback, and addressing performance problems and issues promptly.
\item Attention to Communication: The ability to ensure that information is passed on to others who should be kept informed.
\end{itemize}

\section{The Gaming Industry is the Exception}
For most of software engineering, the demand for engineers exceeds the supply.  In game programming, the supply exceeds the demand.  This means the gaming industry plays by a different set of rules where engineers have much less leverage.
\begin{itemize}
\item Making video games professionally is very different from making games for fun
\item Prepare for a 25\% to 50\% salary cut vs what you could get in other entry level software development jobs.  That is not a typo
\item Game programming is not game design.  Game programmers usually get very little creative input in how the game is made
\item Every job opening gets at least 20 applications with the only project being a platformer made in Unity.  You need a strong understanding of lower level programming in C++ to work in the game industry
\item Be careful who you take advice from.  There are a lot of people who got lucky with their Unity platformer, there is no guaranties you'll be lucky too
\item The gaming industry only gives internships to students from top ranked schools---like USC/UCLA/UCI---or to students with adjacent work experience---with things like C++/graphics/Objective-C/Android-JNI development.  Work experience still counts more than school ranking
\item Most people get in by slogging through QA.  Some move from adjacent software engineering jobs.  Few get in directly by networking directly into game programming jobs
\item Game programming jobs are unstable.  Don't expect your first job to last you a year.  Two years is a long time to work in a single job
\item Layoffs are always around the corner, less than 5\% of project are successful.  Always have 6 months of expenses saved up
\item You may get hired for a project that never gets started.  It's not rare for someone to get hired and then get laid off weeks later because the project they were working on got canceled
\item Specialize.  Specialists get treated better.
\item Just because you made it in does not mean you will last.  The average time in the game industry for an engineer is about 5 years.  Never stop researching to keep your skills sharp
\item Interviews get harder the longer you last in the industry.  Never stop researching
\item Game programming salaries eventually catch up to other software engineering salaries if you have the skills to back it up.  Most people don't.  Never stop researching
\item Crunch happens but it should not be the norm.  If the studio is not making any effort to avoid crunch, get out 
\item Always make sure you can walk away from a job in case you need/have to
\item Game programming skills often don't carry over to in demand positions (fullstack, frontend, backend, DevOps, etc).  It may be difficult to leave if you don't want to be in the gaming industry anymore
\item Backend programming is just as important as frontend programming.  It's just as important to be able to develop a robust game server as to develop a slick game interface
\end{itemize}
\end{document}
